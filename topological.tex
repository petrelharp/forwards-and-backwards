\documentclass[12pt]{article}
\usepackage{fullpage}
\usepackage{amssymb}
\usepackage{ntheorem}
\usepackage{graphics} 
\usepackage{amsmath}
\usepackage[color=yellow]{todonotes}
\usepackage{url}
\usepackage{xcolor}
\usepackage{longtable}
\usepackage[hidelinks]{hyperref}
\usepackage{bbm}

\usepackage{natbib}

\newenvironment {proof}{{\noindent\bf Proof }}{\hfill $\Box$ \medskip}

\newtheorem{theorem}{Theorem}[section]
\newtheorem{lemma}[theorem]{Lemma}
\newtheorem{condition}[theorem]{Condition}
\newtheorem{proposition}[theorem]{Proposition}
\newtheorem{remark}[theorem]{Remark}
\newtheorem{hypothesis}[theorem]{Hypothesis}
\newtheorem{corollary}[theorem]{Corollary}
\newtheorem{example}[theorem]{Example}
\newtheorem{definition}[theorem]{Definition}
\newtheorem{notation}[theorem]{Notation}
\newtheorem{assumptions}[theorem]{Assumptions}
\newtheorem{terminology}[theorem]{Terminology}
\newtheorem*{terminology-non}[theorem]{Terminology}

\renewcommand{\theequation}{\arabic{section}.\arabic{equation}}
\def \non{{\nonumber}}
\def \hat{\widehat}
\def \tilde{\widetilde}
\def \bar{\overline}
\newcommand{\IP}{\mathbb P}
\newcommand{\IQ}{\mathbb Q}
\newcommand{\IE}{\mathbb E}
\newcommand{\IR}{\mathbb R}
\newcommand{\IZ}{\mathbb Z}
\newcommand{\IN}{\mathbb N}
\newcommand{\IT}{\mathbb T}
\newcommand{\IC}{\mathbb C}
\newcommand{\ind}{\mathbf{1}}
\newcommand{\bigO}{\mathcal{O}}
\newcommand{\grad}{\nabla}
\newcommand{\dif}{\mathrm{d}\,}

%%%%%%% notation
\newcommand{\DG}{\mathcal{B}}  % generator of the dispersal process
\newcommand{\DD}{\mathcal{D}}  % the second-order part of the generator of dispersal
\newcommand{\meanq}{\vec b}    % mean of dispersal, times theta
\newcommand{\covq}{C}     % covariance matrix of dispersal, times theta
\newcommand{\kernel}{\rho}  % interaction kernels
\newcommand{\smooth}[1]{\kernel_{#1} \! * \!}  % convolution by the interaction kernel
\newcommand{\wavespeed}{\mathfrak{c}}    % speed (vector) of a wave
\newcommand{\Lgen}{\mathcal{L}}    % generator of a lineage
\newcommand{\Pgen}{\mathcal{P}}    % generator of the population process
\newcommand{\lp}{\xi}              % process with levels
\newcommand{\labelspace}{\mathcal{I}} % space of labels
\newcommand{\concat}{\oplus}   % concatenation of labels
\newcommand{\measures}{\mathcal{M}_F(\IR^d)} % finite measures on Rd
\newcommand{\cmeasures}{\mathcal{M}_F(\overline{\IR^d})} % finite measures on compactified Rd
\newcommand{\lpmeasures}{\mathcal{M}(\overline{\IR^d} \times [0,\infty))} % locally finite measures on space x level


\newcommand{\plr}[1]{\todo[inline]{Peter: #1}}
\newcommand{\comment}[1]{{\color{blue} \it #1}}

\begin{document}



%%%%%%%%%%%%%%%%%%%%%%%%%%%%%%%%%%%%%%%%%%%%%%
\section{Topological Results on Lookdown Representations}
\label{sec: Topologies on Lookdown}
\subsection{Topology for convergence of lookdown representations}

Recall that
the lookdown representation has a spatial component and a level component.
Specifically, for fixed $N>0$ and $t>0$,
the lookdown representation $\xi^N(t)$ 
is a finite measure on $\overline{\mathbb{R}^d} \times [0,N]$.
Furthermore, the level of a particle in our lookdown representation can shoot to infinity in finite time.
Therefore, to characterise convergence of $\xi^N(t)$ as $N \to \infty$,
we need to impose an appropriate vague topology on $\mathcal{M}(\overline{\mathbb{R}^d} \times [0,\infty))$,
the space of finite measures on $\overline{\mathbb{R}^d} \times [0,\infty)$.
Similarly, to characterise convergence of the level process of individual lineages, we consider the vague topology on $[0,\infty)$.



We now refer to the vague topology on measure spaces introduced in P564, \S A2 in \cite{kallenberg1997foundations}.
Let $S$ be a locally compact Polish space and $\hat{S}$ be the class of bounded sets in $S$.
We consider the family $C^{+}_c(S)$ of positive continuous functions with compact support. It is known that $C^{+}_c$ is separable in the uniform metric.
We now consider $\mathcal{M}(S)$, the class of measures on $S$ that are locally finite.
The \textit{vague topology} on $\mathcal{M}(S)$ is generated by the the maps $\mu \to \mu f := \int f d \mu$, where $f\in C^{+}_{c}(S)$.
In other words, convergence of $\mu^n \to \mu$ in the vague topology holds if 
$$\int f d\mu^n \to \int f d\mu$$
for all $f\in C^{+}_{c}(S)$.

We now introduce an important theorem on compactness in the vague topology.
\begin{theorem}[Theorem A.23 in \cite{kallenberg1997foundations}]
For any locally compact, second-countable Hausdorff Space $S$, 
we have
\begin{itemize}
\item $\mathcal{M}(S)$ is Polish in the vague topology,
\item a set $A\subset \mathcal{M}(S)$ is vaguely relatively compact iff $\sup_{\mu\in A} \mu f < \infty$ for all $f \in C^{+}_{c}$
\item if $\mu_n \to \mu$ in the vague topology and $B \in \hat{S}$ with $\mu \partial B = 0$, then $\mu_n B \to \mu B$
\end{itemize}
\end{theorem}

To characterise convergence of lookdown representations,
we take $S = \overline{\mathbb{R}^d} \times [0,\infty)$.
We now identify compact sets under the vague topologies.

\begin{lemma}
\label{lem: vague compact sets}
For any fixed $N.0, K>0$, the set 
$$\Gamma_{N,K} := \left\{\mu \in \mathcal{M}(\overline{\mathbb{R}^d} \times [0,\infty)):  \frac{1}{N}\mu(\overline{\mathbb{R}^d}\times [0,N]) \leq K \right\}$$
is vaguely compact in $\mathcal{M}(S)$.
\end{lemma}

\begin{proof}
It suffices to show that $\Gamma_K$ is vaguely relatively compact and vaguely closed.
For any fixed $f \in C^{+}_c$, $f$ is bounded above uniformly by $||f||_{\infty}$.
Therefore, 
we have that $ \sup_{\mu \in  \Gamma_K} \mu f \leq  ||f||_{\infty} \mu(\overline{\mathbb{R}^d}\times [0,N])  \leq ||f|| K N < \infty$,
so $\Gamma_{N,K}$ is vaguely relatively compact.

Now it suffices to prove that $\Gamma_K$ is closed under the vague topology.
Consider a vaguely converging sequence $(\mu_n)_{n \geq 1}$ in $\Gamma_{N,K}$
with limit $\mu_n \to \mu \in \mathcal{M}(S)$.
We pick some test functions $\{f_m : m \geq 0 \}$ monotonically increasing to $1_{\overline{\mathbb{R}^d}\times[0,N]}(x)$.

Since for each $n>0$ and $m > 0$,
$\frac{1}{N}\langle f_m , \mu_n \rangle < K$,
by Dominating Convergence Theorem,
\begin{equation}
\frac{1}{N}\mu(\overline{\mathbb{R}^d}\times [0,N])
= \lim_{m \to \infty} \frac{1}{N}\langle f_m , \mu \rangle
=\lim_{m \to \infty} \lim_{n \to \infty} \frac{1}{N}\langle f_m , \mu_n \rangle
\leq K,
\end{equation}
and so $\mu \in \Gamma_{N,K}$.
Therefore $\Gamma_{N,K}$ is vaguely compact. 
\end{proof}


We now include another theorem that is very relevant to establishing tightness of $\mathcal{M}(S)$-valued processes in the Skorokhod topology. It can be found as Theorem 3.9.1 in \cite{EK}.

\begin{theorem}[\cite{EK} Theorem 3.9.1]
\label{teo: EK tightness theorem}
Let $(S,r)$ be complete and separable, and let $\{X_{\alpha}\}$ be a family of processes with sample paths in $D_{S}([0,\infty))$.
Suppose that the compact containment condition holds. That is, for every $\eta > 0$ and $T > 0$,
there exists a compact set $\Gamma_{\eta, T} \subset E$ for which 
\begin{equation}
\inf_{\alpha} P \{ X_{\alpha}(t) \in \Gamma_{\eta,T}  ~~ \text{for } 0\leq t \leq T\} \geq 1- \eta
\end{equation}
Let $H$ be a dense subset of $\overline{C}(E)$ (bounded continuous functions on $E$) in the topology of uniform convergence on compact sets. 
Then $\{X_{\alpha}\}$ is relatively compact if and only if $\{f(X_{\alpha})\}$ is relatively compact (as a family of processes with sample paths in $\mathcal{D}_{\mathbb{R}}([0,\infty))$) for each $f \in H$.
\end{theorem}


To apply Theorem \ref{teo: EK tightness theorem},
we need to identify dense subset of $\overline{C}(\mathcal{M}(\overline{\mathbb{R}^d} \times [0,\infty)))$.
Since the space $\mathcal{M}(\overline{\mathbb{R}^d} \times [0,\infty))$ is not compact but locally compact,
we will apply Stone-Weierstrass theorem for locally compact space.

\begin{lemma}[Stone-Weierstrass Theorem for locally compact space]
\label{lem: SW locally compact}
Let $X$ be a locally compact Hausdorff space, and consider the set  $C_{\infty}(X, \mathbb{R})$ of real-valued continuous functions which vanishes at infinity. 
Let $\mathcal{A}$ be a subalgebra in $C_{\infty}(S)$. Then $\mathcal{A}$ is dense in $C_{\infty}(X, \mathbb{R})$ if and only if it separates points and vanishes nowhere.
\end{lemma}

With this version of Stone-Weierstrass Theorem, we can prove the following lemma.
\begin{lemma}
\label{lem: uniform compact density}
We write $\mathcal{A}$ for the algebra (closed under multiplication and addition) generated by 
the collection of functions $\xi \mapsto \langle f, \xi \rangle$, where $ f \in C_c(\overline{\mathbb{R}^d} \times [0,\infty))$ is differentiable in $u$ and smooth in $x$ with bounded support.
Then $\mathcal{A}$ is dense in $C(\mathcal{M}(\overline{\mathbb{R}^d} \times [0,\infty)))$ under the topology of uniform convergence on compact sets.
\end{lemma}
\begin{proof}
Since $\mathcal{A}$ contains the functions $\langle \rho_{\varepsilon}*\mathbbm{1}_{A} , \mu \rangle$,
where $\rho$ is some mollifier and $A$ is a compact set, it clearly
separates points and vanishes nowhere.
Therefore, $\mathcal{A}$ is dense in $C_{\infty}(\mathcal{M}(\overline{\mathbb{R}^d} \times [0,\infty)), \mathbb{R})$.

Furthermore, under the topology of uniform convergence on compact sets, 
$C_{\infty}(\mathcal{M}(\overline{\mathbb{R}^d} \times [0,\infty)), \mathbb{R})$ 
contains the mollified indicator functions $\langle \rho_{\varepsilon}*\mathbbm{1}_{A} , \mu \rangle$,
which are dense in $C(\mathcal{M}(\overline{\mathbb{R}^d} \times [0,\infty)))$.
\end{proof}

Finally, we include another technical lemma on the products of real-valued stochastic processes. 
\begin{lemma}
\label{teo: product tightness}
Let $\{(X^n_t)_{t \geq 0}: n \geq 0\}, \{(Y^m_t)_{t \geq 0}: m \geq 0\}$ be two tight sequences of real-valued process under the Skorokhod topology.
Then the sequence of processes $\{(X^n_tY^n_t)_{t \geq 0}: n \geq 1\}$ is tight under the Skorokhod topology.
\end{lemma}

\begin{proof}
We apply the Aldous Criterion of tightness on the real-valued sequences
$\{(X^n_tY^n_t)_{t \geq 0}: n=1,2,.. \}$. 
It suffices to show that
for fixed sequences of stopping time $\tau_n$ bounded above by $T$,
for each $\varepsilon>0$,
there exists $\delta > 0$ and $n_0$ such that
\begin{equation}
\sup_{n \geq n_0}\sup_{\theta \in [0, \delta]} \mathbb{P}\left[|X^n_{\tau_n+\theta}Y^n_{\tau_n+\theta} - X^n_{\tau_n}Y^n_{\tau_n} | > \varepsilon \right] \leq \varepsilon.
\end{equation}

First note that 
\begin{equation}
|X^n_{\tau_n+\theta}Y^n_{\tau_n+\theta} - X^n_{\tau_n}Y^n_{\tau_n} |
= |X^n_{\tau_n+\theta}(Y^n_{\tau_n+\theta}-Y^n_{\tau_n}) +Y^n_{\tau_n}( X^n_{\tau_n+\theta}- X^n_{\tau_n})|,
\end{equation}
therefore
\begin{equation}
\begin{aligned}
& \mathbb{P}\left[|X^n_{\tau_n+\theta}(Y^n_{\tau_n+\theta}-Y^n_{\tau_n}) +Y^n_{\tau_n}( X^n_{\tau_n+\theta}- X^n_{\tau_n})| > \varepsilon \right]\\
 \leq &
\mathbb{P}\left[\max\{|X^n_{\tau_n+\theta}(Y^n_{\tau_n+\theta}-Y^n_{\tau_n})|, | Y^n_{\tau_n}( X^n_{\tau_n+\theta}- X^n_{\tau_n}) |\}> \varepsilon/2 \right]\\
\leq & \mathbb{P}\left[|X^n_{\tau_n+\theta}(Y^n_{\tau_n+\theta}-Y^n_{\tau_n})|> \varepsilon/2 \right] + \mathbb{P}\left[ | Y^n_{\tau_n}( X^n_{\tau_n+\theta}- X^n_{\tau_n}) |> \varepsilon/2 \right]
\end{aligned}
\end{equation}

By Corollary 3.7.1 in \cite{EK}, both $\{(X^n_t)_{t \geq 0}: n \geq 0\}, \{(Y^m_t)_{t \geq 0}: m \geq 0\}$ satisfies the compact containment condition, i.e. so for any $\varepsilon'>0$ and fixed time $T> 0$, there exists $K_{\varepsilon',T}>0$ such that 

\begin{equation}
 \limsup_{n \to \infty} \mathbb{P}\left\{ \sup_{t\in [0,T]}X^n_t > K_{\varepsilon',T} \right\} < \epsilon',
\end{equation}
and 
\begin{equation}
 \limsup_{n \to \infty} \mathbb{P}\left\{ \sup_{t\in [0,T]}Y^n_t > K_{\varepsilon',T} \right\} < \epsilon'.
\end{equation}

Conditioning on the event $\left\{ \sup_{t\in [0,T]}Y^n_t > K_{\varepsilon',T} \right\}$ and $\left\{ \sup_{t\in [0,T]}Y^n_t > K_{\varepsilon',T} \right\},$

\begin{equation}
\begin{aligned}
& \mathbb{P}\left[|X^n_{\tau_n+\theta}(Y^n_{\tau_n+\theta}-Y^n_{\tau_n})|> \varepsilon/2 \right] + \mathbb{P}\left[ | Y^n_{\tau_n}( X^n_{\tau_n+\theta}- X^n_{\tau_n}) |> \varepsilon/2 \right]\\
\leq & \mathbb{P}\left[|(Y^n_{\tau_n+\theta}-Y^n_{\tau_n})|> \frac{\varepsilon}{2K_{\varepsilon',T}}\right] + \mathbb{P}\left[ | Y^n_{\tau_n}( X^n_{\tau_n+\theta}- X^n_{\tau_n}) |> \frac{\varepsilon}{2K_{\varepsilon',T}}\right] + 2 \varepsilon'\\
\leq & \frac{\varepsilon}{K_{\varepsilon',T}}+2 \varepsilon',
\end{aligned}
\end{equation}
and by tightness of $(X^n_t, Y^n_t)_{t \geq 0}$,
there exists $\delta > 0$ and $n \geq n_0$ such that the last term is smaller than $\varepsilon$ for 
for all $\theta < \delta$ and $n\geq n_0$.
\end{proof}


When we work with lookdown representations, we work on test functions of the form
\begin{equation}
F_g(\xi) = \prod_{(x,u) \in \xi} g(x,u),
\end{equation}
where $g(x,u)$ is differentiable in $u$, smooth in $x$.
and satisfies the property that $0\leq g(x,u) \leq 1$,
and that there exists $u^*$ such that for all $u > u^*$,
$g(x, u) =1$.

In fact, there is a one-to-one correspondence between $C^{+}_c(S)$
and test functions of the form $F_g$.
For all $f \in C^{+}_c(S)$,
if we write $g = e^{-f}$,
then
\begin{equation}
F_g(\xi)= e^{-\langle f, \xi \rangle}.
\end{equation}

Furthermore,
we know that for any $x_1, x_2 \in [0,N]$,
$$|x_1-x_2|\geq |e^{-x_1}-e^{-x_2}| \geq e^{-N}|x_1-x_2|,$$
so we can control $\langle f, \xi^n_t \rangle $ by controlling $F_g(\xi^n_t)$ as long as 
there exists some $N > 0$ such that $\langle f, \xi^n_t \rangle < N$.

With the above observation,
we have the following lemma.

\begin{lemma}
We consider a sequence $(\xi^n_t)_{t\geq 0}$ of $\mathcal{M}(\overline{\mathbb{R}^d} \times [0,\infty))$-valued cadlag processes.

Assume that the sequence $(\xi^n_t)_{t\geq 0}$ satisfies the compact containment condition. That is,
for all $\varepsilon > 0$ and fixed time $T > 0$, there exists $N, K> 0$ such that 
\begin{equation}
\liminf_{n \to \infty} \mathbb{P}\{ \xi^n_t \in \Gamma_{N,K} \text{ for all } t\in [0,T]\} > 1-\varepsilon,
\end{equation}
where $\Gamma_{N,K}$ is the compact set in Lemma \ref{lem: vague compact sets}

Assume that for all test functions of the form $F_g$,
the sequence $\{(F_g(\xi^n_t))_{t \geq}: n=1,2,...\}$ of real-valued cadlag processes are relatively compact. 

Then  $(\xi^n_t )_{t\geq 0}$ is relatively compact in $\mathcal{D}_{\mathcal{M}(\overline{\mathbb{R}^d} \times[0,\infty))}([0,T])$.
\end{lemma}

\begin{proof}
By Lemma  \ref{lem: uniform compact density}, \ref{teo: product tightness}, and Theorem 3.9.1 in \cite{EK},
 $(\xi^n_t )_{t\geq 0}$ is relatively compact in $\mathcal{D}_{\mathcal{M}(\overline{\mathbb{R}^d} \times[0,\infty))}([0,T])$
 if we can prove tightness on the real-valued sequences
$\{(f(\xi^n_t))_{t \geq 0}: n=1,2,.. \}$.

Applying the Aldous Criterion, it suffices to show that
for fixed sequences of stopping time $\tau_n$ bounded above by $T$,
for each $\varepsilon>0$,
there exists $\delta > 0$ and $n_0$ such that
\begin{equation}
\sup_{n \geq n_0}\sup_{\theta \in [0, \delta]} \mathbb{P}\left[|\langle f, \xi^n_{\tau_n+\theta}\rangle - \langle f, \xi^n_{\tau_n}\rangle | > \varepsilon \right] \leq \varepsilon.
\end{equation}

First we pick $N,M>0$ such that 
 $$\sup_{n \geq M} \mathbb{P}\left\{ \sup_{t\in [0,T]}\langle f, \xi^n_t \rangle > N \right\} < \epsilon /2 .$$
Then, we can partition the event $\{|\langle f, \xi^n_{\tau_n+\theta}\rangle - \langle f, \xi^n_{\tau_n}\rangle | > \varepsilon\}$ into two subsets according to the event 
$ \sup_{t\in [0,T]}\langle f, \xi^n_t \rangle > N$.

By the compact containment condition, 
we know that for any $f \in C^{+}_{c}(\overline{\mathbb{R}^d} \times [0,\infty))$ and $\epsilon > 0$, there exists $N>0$ (dependent on $f$ and $\epsilon$), 
such that 
\begin{equation}
\label{eq: vague compact containment condition}
 \limsup_{n \to \infty} \mathbb{P}\left\{ \sup_{t\in [0,T]}\langle f, \xi^n_t \rangle > N \right\} < \epsilon.
\end{equation}

Therefore, for all $n>M$,
\begin{equation}
\begin{aligned}
 &\mathbb{P}\left[|\langle f, \xi^n_{\tau_n+\theta}\rangle - \langle f, \xi^n_{\tau_n}\rangle | > \varepsilon \right]\\
 =& \mathbb{P}\left[|\langle f, \xi^n_{\tau_n+\theta}\rangle - \langle f, \xi^n_{\tau_n}\rangle | > \varepsilon ,  \sup_{t\in [0,T]}\langle f, \xi^n_t \rangle > N \right]\\
 & + \mathbb{P}\left[|\langle f, \xi^n_{\tau_n+\theta}\rangle - \langle f, \xi^n_{\tau_n}\rangle | > \varepsilon,  \sup_{t\in [0,T]}\langle f, \xi^n_t \rangle \leq N \right]\\
< & \varepsilon/2 + \mathbb{P}\left[|e^{-\langle f, \xi^n_{\tau_n+\theta}\rangle} - e^{-\langle f, \xi^n_{\tau_n}\rangle }| > e^{-N}\varepsilon,  \sup_{t\in [0,T]}\langle f, \xi^n_t \rangle \leq N \right]\\
< & \varepsilon/2 + \mathbb{P}\left[|F_g(\xi^n_{\tau_n+\theta})-F_g(\xi^n_{\tau_n})| > e^{-N}\varepsilon\right].
\end{aligned}
\end{equation}

Finally, by tightness of the processes $\{F_g(\xi^n_t): n = 1,2,...\}$, 
we can pick large enough $n_0 > M$ and $\delta > 0$
such that 
$$\sup_{n \geq n_0}\sup_{\theta \in [0,\delta]}\mathbb{P}\left[|F_g(\xi^n_{\tau_n+\theta})-F_g(\xi^n_{\tau_n})| > e^{-N}\varepsilon\right]< e^{-N}\varepsilon,$$
so
\begin{equation}
\sup_{n \geq n_0}\sup_{\theta \in [0, \delta]} \mathbb{P}\left[|\langle f, \xi^n_{\tau_n+\theta}\rangle - \langle f, \xi^n_{\tau_n}\rangle | > \varepsilon \right] \leq \varepsilon / 2 + e^{-N} \varepsilon < \varepsilon.
\end{equation}

We have proved relative compactness of $\{(\langle f, \xi^n_t \rangle )_{t \geq 0}: n=1,2,.. \}$, and thus tightness of 
 $\{(\xi^n_t )_{t \geq 0}: n=1,2,.. \}.$
\end{proof}



\bibliographystyle{plainnat}
\bibliography{refs.bib,plr_refs.bib}


\end{document}
