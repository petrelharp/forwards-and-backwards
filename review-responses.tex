%%%%%%
%%
%%  Don't reorder the reviewer points; that'll mess up the automatic referencing!
%%
%%%%%

\begin{minipage}[b]{2.5in}
  Resubmission Cover Letter \\
  {\it Theoretical Population Biology}
\end{minipage}
\hfill
\begin{minipage}[b]{2.5in}
    Peter Ralph \\
  \today
\end{minipage}
 
\vskip 2em
 
\noindent
{\bf To the Editor(s) -- }
 
\vskip 1em

We are writing to submit a revised version of our manuscript,
ETCETERA

\vskip 2em

\noindent \hspace{4em}
\begin{minipage}{3in}
\noindent
{\bf Sincerely,}

\vskip 2em

{\bf 
Peter Ralph
}\\
on behalf of \\
{\bf 
Alison Etheridge,
Tom Kurtz,
Ian Letter,
and Terence Tsui Ho Lung.
}\\
\end{minipage}

\vskip 4em

\pagebreak

%%%%%%%%%%%%%%
\section*{AE comments:}


%%%%%%%%%%%%%%
\reviewersection{1}

\begin{quote}
The authors introduced a class of spatial population models in which individuals have locations
in the continuum space $\IR^d$. The population dynamics is controlled by three quantities: birth,
establishment and death, each of which can depend on the spatial location and local population
density. For a fixed time $t$, the population can be represented by a point measure $\eta_t^N$ in which
each individual is assigned a mass $1/N$, where $N$ is a parameter such that initially there are $O(N)$
individuals per unit volume in $\IR^d$. The population dynamics is represented by a continuous-time
Markov process $(\eta_t^N)_{t\ge0}$. The key feature of this model is that it distinguishes between juveniles
and adults. The authors obtained three different scaling limits for the model $(\eta_t^N)_{t\ge0}$ as $N \to \infty$,
which are super-processes, non-local differential equation and reaction-diffusion type PDE. The
author also obtained a description of the dynamics of a single lineage as $N \to \infty$, when the
limiting population dynamics is deterministic.

A main contribution of this paper is the rigorous derivation of lineage dynamics through the
lookdown construction. The convergence of the population dynamics to super-processes, non-
local PDE and PDE are also significant, as they inform us of not only the limiting object that are
expected to be robust against model perturbation, but also of the corresponding scaling regimes
and parameter combinations for model reduction. The one-step convergence presents a usual technical
challenge about identifying the limit. The authors overcame the challenge masterfully by
establishing continuity estimates that guarantee good agreement for the nonlinear terms between
the limiting PDE and the stochastic model, under high mixing rate and a high density assumption.
Overall, this is a very high quality paper that contains many interesting and novel results to merit
publication in EJP, after minor revision.
\end{quote}

Thanks ETCETERA

\subsection*{Major comments}

\begin{point}{\revref}
The first paragraph seems sloppy. For example, this expression [for $\mu_\theta(x)$] is non-negative for
what range of values for $(x, \eta)$? Why ``anyway true for sufficiently large $\theta$''? Perhaps this
assumption should be labeled and called upon whenever is it being referred to.
\end{point}

\reply{
}

\begin{point}{\revref}
Definition \ref{def:lineage_generator}, Proposition 2.23: I think these may need to be rewritten/rephrased. Up
to Proposition \ref{thm:lineages}, it is not clear what ``that individual'' is in the limiting model. Therefore,
it is not clear what $L_s$ is. While the definition of the process $L^N$ maybe clear, what is the
precise definition of $L_s$ ? I think these need to be defined through the finite-$N$ models. It may
be good to explain why the process $L$ is Markovian, or at least point out that it is Markovian,
before talking about semigroup. Probably Section \ref{sec:lineage_motion} will need to be modified accordingly.
\end{point}

\reply{
    Add sentence saying that this is well-defined because we take the finite-$N$ limit.
    Markovian because it is conditional on the population process, which is deterministic...
}

\begin{point}{\revref}
The two conditions in equation \eqref{conditions for one step} are crucial for the one-step convergence.
It would be great to mention exactly where each of them are used in Section \ref{subsec:one step convergence proof}.
A brief explanation about these conditions would also be nice. For example, it seems the first (high
migration/mixing rate) condition in \eqref{conditions for one step},
but not the second (high density) condition, is used
to guarantees that $\psi_t^{\epsilon,x}$ is close to a heat kernel?
\end{point}

\reply{
    Add discussion of that here.
    (in particular the one condition comes from the crucial lemma 7.2)
}

\subsection*{Minor comments}

\begin{point}{\revref}
\eqref{eqn:mu_defn}: The authors mentioned ``define the death rate'' and ``where this equation defines F'',
which is confusing. What is being defined here? Please clarify. It seems to me that one has
to first choose $F$, then define $µ_\theta(x)$ as in \eqref{eqn:mu_defn}.
\end{point}

\reply{
}

\begin{point}{\revref}
% P7, paragraph after Remark 2.3:
``and once there establishes...'' It maybe better to split this
long sentence into two complete sentences.
\end{point}

\reply{
}

\begin{point}{\revref}
% P7, paragraph after Remark 2.3:
$\meanq(x)$ and $\covq(x)$ did not appear before. Perhaps mention they
are given/chosen, or simply delete them here, and mention them in Definition \ref{def:dispersal_generator}.
\end{point}

\reply{
}

\begin{point}{\revref}
Remark \ref{ancestral lineages: first guess}
Why use $m$ in the superscript of $q_\theta^m(x, \eta, dy)$?
\end{point}

\reply{
}

\begin{point}{}
Assumptions \ref{def:model_setup}: Should it be ``Assumption 2.8'' instead? Also, $\alpha$ in condition 1 was
used in $\alpha = lim_{N \to \infty} \theta(N)/N$ on Page 8, Figure 1, etc.
\end{point}

\reply{
}

\begin{point}{\revref}
% P9 last line:
    Should the ``$C$'' on the right hand side of the equal sign [in the equation for $C_f$] be deleted ?
\end{point}

\reply{
}

\begin{point}{\revref}
% P10 paragraph below Lemma 2.9:
    Why ``no longer suffices'', and why this is true? . Please
explain. I don’t seem to find this explained in any remark before this.
\end{point}

\reply{
}

\begin{point}{}
Figure 1: What is $N$, the time and other parameters (e.g. $\rho_r$ , $\rho_\gamma$) in the simulations? Same
for other figures. Please at least specify the range of values.
\end{point}

\reply{
}

\begin{point}{\revref}
% P12 paragraph after Corollary 2.11:
A more precise reference, or explanation for why
uniqueness holds for this special case would be helpful, as the references you ave are all
quite extensive.
\end{point}

\reply{
}

\begin{point}{}
Proposition \ref{nonlocalPME to PME}:
    It seems the constant $C$ is not needed. Perhaps just say the supremum
of the first displayed expression over $\epsilon$ finite?
\end{point}

\reply{
}

\begin{point}{}
Remark \ref{remark_on_nonlocal_to_local}: Can you explain a bit more why it will be necessary to retain averaging of r?
\end{point}

\reply{
    Since you need to apply the Laplacian to it.
}

\begin{point}{\revref}
% P16: Second paragraph:
The notation $L_t$ seems to first appear here and was not yet defined.
\end{point}

\reply{
    (Note: they said ``P16: Second paragraph''; I think they just missed it on the previous page.)
}

\begin{point}{}
% P19-20:
    The equations for the lineage dynamics for the three examples [i.e., Fisher-KPP \revref, Allen-Cahn, and Porous Medium] are interesting.
Which is/are new in this paper, and which was/were known before (and related references)?
\end{point}

\reply{
    The PME is new;
    probably the others are in Birzu/Korolev/Hallaschek? 
    Also maybe see Roques et al 2008ish PNAS?
    It reduces to BM in a potential:
    find other references for this (i.e., the $\grad \log$...)
}

\begin{point}{}
This maybe out of the scope of this paper, but it is related. As the authors pointed out, two
lineages probably will not coalesce in the limiting model. However, they will coalesce in
the finite-$N$ model, though it may take a very long time. I wonder if the authors have a
feeling about whether the description of the lineage dynamics in the limit still be a good
approximation for that of the finite model for such a long time.
\end{point}

\reply{
    Just say we don't know, but:
    see Pennington and Etheridge on bistable waves
    where fluctuations in the wave happen on long time scales
}

\begin{point}{}
The subtitle of section \ref{sec:nonunique_lineage} does not match the actual content in that section. Please
update the subtitle. Also, can you give a reference for the claim in the last sentence in this
section \revref?
\end{point}

\reply{
}

\begin{point}{\revref}
specify ``this process'' in the line below the first displayed equation in
Section 5.1.
\end{point}

\reply{
}

\begin{point}{\revref}
Paragraph starting with ``Levels never cross below 0'': Should it be instead ``Levels never
go above $N$''? If so, what particles are regarded as dead? Perhaps those exit the interval
$[0, N]$? Note that equation \eqref{eqn:dot_u} suggests that levels can now go above $N$.
\end{point}

\reply{
}

\begin{point}{\revref}
``Recall from equation~\eqref{eqn:mu_defn}''
does not seem correct, as there is a max in \eqref{eqn:mu_defn}. Please
correct this.
\end{point}

\reply{
}

\begin{point}{\revref}
Please give a precise reference/explanation to why uniqueness holds for the martingale problem for finite $N$.
\end{point}

\reply{
    ``Since this is a finite-rate jump process...''
}

\begin{point}{\revref}
% P32 third line in Section 5.3:
    Pointwise in time?
\end{point}

\reply{
    $\omega$
}

\begin{point}{\revref}
Proposition \ref{prop:limiting_construction}: In ``For any solution to the equation above'', what is a ``solution''? Is it
a solution for all time? Also, be specific to ``the equations above'' by labeling and referring
to the precise equations including the unlabeled displayed equation below \eqref{eqn:limiting_construction}.
\end{point}

\reply{
    Let $X$, $U$ be solutions to 5.20 on $[0,T]$... then $\eta$ is a solution on $[0,T]$...
}

\begin{point}{\revref}
The word ``Proof'' is missing in ``Proof of Proposition 5.6''. This word is also
missing in other places in Sections 6-8.
\end{point}

\reply{
}

\begin{point}{}
``Proof of Proposition \ref{prop:limiting_construction}'': Do you also need the unlabeled displayed equation below
\eqref{eqn:limiting_construction}? The solution is for all time.
\end{point}

\reply{
}

\begin{point}{\revref}
Be specific about ``these results'' in ``Proof of the remainder of these results...'' Also, at the
very beginning of Remark \ref{remark_on_branching}, Be specific about what ``The process'' is.
\end{point}

\reply{
}

\begin{point}{\revref}
``Condition 1'' here is that for Lemma 2.9, not that for Assumptions 2.8. Please specify.
\end{point}

\reply{
}

\begin{point}{\revref}
Perhaps give a reference to the ``familiar pattern''.
\end{point}

\reply{
    Alison's superprocess book, section 1.4.
}

\begin{point}{\revref}
% P40 first displayed equation:
Mention $M_t$ is a martingale for all such test functions.
\end{point}

\reply{
    i.e., add ``for all such test functions''
}

\begin{point}{\revref}
%    P48:
    The notation $\mathcal{O}(\epsilon)$ seems to first appear here. Please explain its meaning. Do they
depend on x?
\end{point}

\reply{
    Alison TODO
}

\begin{point}{\revref}
% P54: near the bottom:
    Can you be more specific what in paper \citet{barlow/jacka/yor:1986} is being used here?
\end{point}

\reply{
    that we can bound the expectation of the quadratic variation by a constant times the expectation of the angle bracket process
    (AS BEFORE)
}

\begin{point}{\revref}
% P60:
The notation $\mathcal{O}(\delta)$ seems to first appear here. Please explain its meaning. Do they
depend on x?
\end{point}

\reply{
    Point out we're using 7.2 with $u=0$ and our assumptions about initial conditions.
    (and maybe write that it's less than a constant times $\delta$;)
}

\begin{point}{\revref}
% P65: Remark 8.1:
measure-values processes $\to$ measure-valued processes
\end{point}

\reply{
    Done. \revref
}

\begin{point}{}
Section \ref{sec:lineages_proof} may need to be modified also based on the second point in ``Major
comments'' above.
\end{point}

\reply{
}

